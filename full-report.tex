\documentclass{itkmitlcoop}

\usepackage{afterpage}
\usepackage{graphicx,amsmath,latexsym,amssymb,amsthm}
\usepackage{indentfirst}
\usepackage{cite}
\usepackage{float}
\usepackage{wrapfig}
\usepackage{eso-pic} 
\usepackage[labelfont=bf]{caption}
\graphicspath{ {images/} }

\makeatletter
% \patchcmd{<cmd>}{<search>}{<replace>}{<succes>}{<failure>}
\patchcmd{\@chapter}{\addtocontents{lof}{\protect\addvspace{10\p@}}}{}{}{}% LoF
\patchcmd{\@chapter}{\addtocontents{lot}{\protect\addvspace{10\p@}}}{}{}{}% LoT
\makeatother

% Your thesis title (THAI)
\newcommand{\ThesisTiTle}{ระบบจัดการโฆษณาแบบจำกัดจำนวนการคลิกและการแสดงโฆษณา}
% Your thesis title (ENG)
\newcommand{\ThesisTiTleENG}{ADVERTISEMENT MANAGEMENT SYSTEM BASED ON LIMITED NUMBER OF CLICKS AND IMPRESSIONS}
\newcommand{\ThesisTiTleENGSnakecase}{Advertisement Management System Based on Limited Number of Clicks and Impressions}
% Your name
\newcommand{\AuName}{มาวิน จงไกรรัตนกุล}
% Your name ENG
\newcommand{\AuNameENG}{MAWIN JONGKRIRATTANAKUL}
\newcommand{\AuNameENGSnakecase}{Mawin Jongkrirattanakul}
% Department / Program
\newcommand{\DepartmentENG}{Information Technology}
% Your student ID
\newcommand{\SId}{59070141}
% Your advisor
\newcommand{\Advisor}{รองศาสตราจารย์ ดร.กิติ์สุชาต  พสุภา}
% Your advisor
\newcommand{\AdvisorENG}{Associate Professor Dr. Kitsuchart Pasupa}
% Your advisor employee
\newcommand{\Exami}{คุณ ธนพล เนรัญชร}
% ชื่อสถานประกอบการ
\newcommand{\Company}{บริษัท วงใน มีเดีย จำกัด (สำนักงานใหญ่)}
% ภาคเรียนที่ (in normal letters)
\newcommand{\Sem}{1}
% ปีการศึกษา (in normal letters)
\newcommand{\AcaY}{2562}
% ปีการศึกษา (in normal letters)
\newcommand{\AcaYAD}{2019}
% วันส่งรายงาน
\newcommand{\SubD}{2 ธันวาคม พ.ศ. 2562}
% วันเริ่มทำงาน
\newcommand{\StartDWork}{4 มิถุนายน พ.ศ. 2562}
% วันสุดท้ายของการทำงาน
\newcommand{\EndDWork}{29 พฤศจิกายน พ.ศ. 2562}
% ที่อยู่สถานประกอบการ
\newcommand{\Address}{เลขที่ 8 ถนนสุขุมวิท แขวงพระโขนง เขตคลองเตย จังหวัดกรุงเทพมหานคร \\ รหัสไปรษณีย์ 10110 โทรศัพท์ 0-2821-5788}
% เว็บไซต์สถานประกอบการ
\newcommand{\Website}{https://www.wongnai.com}
% ตำแหน่งานที่ปฏิบัติ
\newcommand{\Position}{Software Engineer (Backend)}

\begin{document}    
    \frontmatter
    \lhead{}\rhead{}\chead{}\lfoot{}\cfoot{\thepage}\rfoot{}
    \makecover    
    \makeinnercover
    \makeengcover
    \makecopyrightcover
    \makeletter
    \makeack{
        \begin{enumerate}
            \item คุณ ธนพล เนรัญชร \quad \quad ตำแหน่ง Technical Director (พนักงานที่ปรึกษา)
            \item คุณ ปาลิตา เตชะนิเวศน์ \quad ตำแหน่ง Software Engineer (Backend) 
        \end{enumerate}
    }
    \makeapproveletter
   
    % Setting margin for page numbering on frontmatter
    \newgeometry{top=1in, bottom=1in, left=1.5in, right=1in, includefoot}
    
    \makeabstract{
    	รายงานการปฏิบัติงานสหกิจศึกษาฉบับนี้กล่าวถึงที่มาและความสำคัญ, รายละเอียด, การออกแบบ และกระบวนการทำงานของระบบจัดการโฆษณาแบบจำกัดจำนวนการคลิกและการแสดงโฆษณา รวมไปถึงลักษณะขั้นตอนการทำงานเพื่อให้ได้มาซึ่งระบบที่สามารถใช้งานได้จริง โดยทางบริษัท วงใน มีเดีย จำกัด (สำนักงานใหญ่) ได้มอบหมายให้ระหว่างการปฏิบัติงานสหกิจศึกษา ระบบจัดการโฆษณาแบบจำกัดจำนวนการคลิกและการแสดงโฆษณา เป็นระบบที่พัฒนาขึ้นมาจากระบบจัดการโฆษณาเดิมที่มีอยู่ จากเดิมที่ระบบสามารถแสดงโฆษณาได้แค่ตามช่วงเวลาที่กำหนดไว้ ระบบใหม่จะสามารถแสดงโฆษณาตามจำนวนการคลิกและจำนวนการแสดงโฆษณาที่กำหนดไว้ได้ หากโฆษณาถูกแสดงหรือมีผู้ใช้คลิกเข้าไปในโฆษณาจนครบตามจำนวนที่กำหนดไว้ ระบบก็จะหยุดแสดงโฆษณาโดยอัตโนมัติ อีกทั้งยังสามารถรายงานผลการโฆษณากลับไปยังลูกค้าได้โดยอัตโนมัติ ระบบที่ถูกพัฒนาขึ้นมาใหม่นั้น จะทำให้ลูกค้าสามารถลงโฆษณาบนเว็บไซต์ wongnai.com และแอปพลิเคชัน Wongnai ได้อย่างคุ้มค่ามากยิ่งขึ้น เนื่องจากวิธีการแสดงโฆษณาแบบดังกล่าว สามารถการันตีได้ว่า โฆษณาของลูกค้ามีผู้ชมจริง ๆ ในช่วงที่โฆษณายังแสดงผลอยู่ และลูกค้าสามารถติดตามผลการโฆษณาได้อย่างต่อเนื่อง อีกทั้งบนเว็บไซต์ wongnai.com และแอปพลิเคชัน Wongnai ก็สามารถจัดการพื้นที่การโฆษณาได้ดียิ่งขึ้น โฆษณาที่มีผู้ชมมากจะถูกหยุดการแสดงผล และนำโฆษณาอื่นมาแสดงแทน ทำให้โฆษณามีเนื้อหาที่หลากหลายมากยิ่งขึ้น
    }

   \makeabstracteng{
       This cooperative education report presents the statement of significance, specification, design, and workflow of the Advertisement Management System Based on Limited Number of Clicks and Impressions including the development process to develop a system that can be used in production which has been assigned by Wongnai Media Co., Ltd during cooperative education. Advertisement Management System Based on Limited Number of Clicks and Impressions is a system that developed from a former advertisement management system which only able to show advertisements for just the specified period. A newer system will be able to show advertisements based on a number of clicks and impressions. When the advertisements' number of clicks or impressions reaches a limit, the system will stop showing advertisements automatically and also report advertising results back to customers automatically. The newly developed system will allow customers to advertise on the Wongnai website and application more cost-effectively due to the above method of advertisements displaying can guarantee that the client's advertisements will reach to the audience while the advertisements are showing and clients can continuously monitor the advertising results. Moreover, Wongnai will be able to better manage the advertising space. Also on the Wongnai website and application can better manage advertising space. The advertisements with a large audience will stop showing and display other advertisements instead Make the ads have more variety of content.
   }

    \newpage
    \addcontentsline{toc}{chapter}{สารบัญ}
    \tableofcontents
    
    \newpage
    \addcontentsline{toc}{chapter}{สารบัญภาพ}
    \listoffigures
    
    % Reset frontmatter page numbering margin, back to original margin from class file
    \restoregeometry

    \mainmatter
    \lhead{}\rhead{\thepage}\chead{}\lfoot{}\cfoot{}\rfoot{}
    
    \chapter{บทนำ}
\label{chapter:introduction}

บริษัท วงใน มีเดีย จำกัด (สำนักงานใหญ่) เป็นองค์กรที่ให้บริการและดูแลเว็บไซต์ wongnai.com และแอปพลิเคชัน Wongnai บนโทรศัพท์มือถือทั้งบนระบบปฏิบัติการ Android และ iOS (ต่อจากนี้จะขอเรียกว่า Wongnai) ซึ่งที่รู้จักกันอย่างดีสำหรับบริการค้นหา-รีวิวร้านอาหารในประเทศไทย และเป็นแอปพลิเคชันแรก ๆ ของประเทศไทยที่ให้บริการในด้านนี้ ในช่วงแรกของ Wongnai นั้นมีจำนวนผู้ใช้งานน้อย แต่เนื่องด้วยการเข้ามาของ Smartphone ทำให้จำนวนผู้งานเพิ่มขึ้นอย่างก้าวกระโดดเป็นอย่างมาก และปัจจุบัน Wongnai นอกจากจะให้บริการค้นหาและรีวิวร้านอาหารแล้ว ยังสามารถค้นหาที่พัก-ที่เที่ยว, ค้นหาสูตรอาหาร หรือแม้กระทั่งสั่งอาหาร Delivery ก็สามารถทำได้

การโฆษณาถือว่าเป็นส่วนสำคัญอย่างยิ่งที่จะทำให้ผู้บริโภคสามารถรับรู้ถึงการมีตัวตนอยู่ของสินค้าและบริการ นอกจากการสร้างสรรค์โฆษณาให้ดูน่าสนใจแล้ว การเลือกตำแหน่งที่จะแสดงโฆษณาก็ถือว่าเป็นสิ่งที่สำคัญ เพื่อให้โฆษณาเข้าถึงกลุ่มเป้าหมายได้มากที่สุด

ปัจจุบัน Wongnai นั้น มีจำนวนผู้ใช้งานเยอะมากถึง 8 ล้านรายต่อเดือน ~\cite{wongnai} เนื้อหาหลักของ Wongnai เองก็เป็นเรื่องเกี่ยวกับอาหาร, ร้านอาหาร และร้านบริการอื่น ๆ เช่น ร้านเสริมสวย, ร้านนวด เป็นต้น Wongnai จึงนับว่าเป็นตัวเลือกที่ดีสำหรับการโฆษณาที่มีเนื้อหาเกี่ยวกับร้านอาหารและร้านบริการ

แต่เดิมแล้ว Wongnai สามารถแสดงร้านที่เป็นโฆษณาได้ตามช่วงเวลาที่ตกลงกับลูกค้าไว้ ซึ่งโฆษณาจะปรากฏอยู่ในตำแหน่งต่าง ๆ ของเว็บไซต์และแอปพลิเคชัน เพื่อเป็นการสร้างความเชื่อมั่นให้กับลูกค้าที่ต้องการจะลงโฆษณากับ Wongnai จึงจำเป็นต้องมีการพัฒนาระบบจัดการโฆษณาแบบใหม่ขึ้นมา โดยระบบนั้นสามารถแสดงโฆษณาโดยจำกัดจำนวนการคลิกและการแสดงโฆษณา ยกตัวอย่างเช่น โฆษณาหนึ่งถูกจำกัดการแสดงไว้ที่ 10,000 ครั้ง หากมีการแสดงโฆษณาครบ 10,000 ครั้งแล้ว ระบบก็จะนำโฆษณาออกโดยอัตโนมัติ หรือ โฆษณาหนึ่งถูกจำกัดการคลิกไว้ที่ 5,000 ครั้ง หากมีผู้ใช้งานคลิกเข้าไปที่โฆษณาครบ 5,000 ครั้งแล้ว ระบบก็จะนำโฆษณาออกโดยอัตโนมัติ ซึ่งระบบเดิมสามารถแสดงโฆษณาได้แค่ตามช่วงเวลาที่กำหนดไว้ การเพิ่มวิธีการแสดงโฆษณาแบบใหม่จะทำให้ลูกค้าจะรู้สึกคุ้มค่ามากขึ้น เนื่องด้วยวิธีการแสดงโฆษณาแบบใหม่สามารถการันตีได้แน่นอนว่าโฆษณาจะถูกแสดงหรือมีผู้ใช้ Wongnai คนอื่นคลิกเข้าไปที่โฆษณาก่อนที่โฆษณาจะถูกนำออกและ Wongnai เองก็จะสามารถจัดสรรพื้นที่ในการโฆษณาได้อย่างมีประสิทธิภาพมากยิ่งขึ้น สามารถแสดงโฆษณาให้มีเนื้อหาที่หลากหลาย เนื่องจากโฆษณาที่ถูกแสดงบ่อยครั้งหรือมีผู้ที่คลิกเข้าไปในโฆษณาเป็นจำนวนมากจะถูกนำออกไป และแทนที่ด้วยโฆษณาอื่น ๆ แทน

\begin{figure}[!h]
	\centering
	\subfigure[]{
		\label{Fig:listingad:web}
		\includegraphics[width=0.9\textwidth]{listing-ad-web.png}  
	}
	\subfigure[]{
		\label{Fig:listingad:ios}
		\includegraphics[width=0.4\textwidth]{listing-ad-ios.jpeg}  
	}
	\caption{ตัวอย่างการโฆษณาร้านในเว็บไซต์ wongnai.com และในแอปพลิเคชัน Wongnai}
	\label{Fig:listingad}
\end{figure}

\section{วัตถุประสงค์การปฏิบัติงาน}
\begin{enumerate}
	\item เพื่อพัฒนาระบบจัดการโฆษณาแบบจำกัดจำนวนการคลิกและการแสดงโฆษณาที่สามารถใช้งานได้จริง
	\item เพื่อเรียนรู้และหาประสบการณ์ใหม่ ๆ เกี่ยวกับงานด้านวิศวกรรมซอฟต์แวร์โดยการลงมือปฏิบัติงานจริง
	\item เพื่อเรียนรู้และปรับตัวเข้ากับสังคมการทำงาน
\end{enumerate}

\section{ประวัติและรายละเอียดบริษัท}
บริษัท วงใน มีเดีย จำกัด (สำนักงานใหญ่) ตั้งอยู่ที่ อาคารทีวัน ชั้น 26, 27 ซอยสุขุมวิท 40 แขวงพระโขนง เขตคลองเตย กรุงเทพมหานคร 10110 เป็นองค์กรที่ให้บริการเว็บไซต์ wongnai.com และแอปพลิเคชัน Wongnai ทั้งบนระบบปฏิบัติการ Android และ iOS ซึ่งเป็นแอปพลิเคชันค้นหาร้านอาหารของประเทศไทยที่มีข้อมูลครอบคลุมทั้งร้านอาหาร, ร้านเสริมสวย, สปา, สูตรอาหาร, โรงแรม, ที่พักและที่เที่ยว โดยมีเป้าหมายหลัก คือ ต้องการที่จะเชื่อมต่อคนไทยเข้ากับสิ่งดี ๆ ทุกอย่างไม่ว่าจะเป็นร้านอาหารร้านเสริมสวยและธุรกิจบริการอื่น ๆ

\begin{figure}[!h]
	\centering
	\includegraphics[width=0.8\textwidth]{wongnai-logo.png}  
	\caption{ตราสัญลักษณ์ของ Wongnai}
	\label{Fig:wongnai-logo}
\end{figure}
    \chapter{รายละเอียดการปฏิบัติงาน}
\label{chapter:related-theory}

เริ่มสหกิจศึกษาโดยปฏิบัติงานที่ บริษัท วงใน มีเดีย จำกัด (สำนักงานใหญ่) ตั้งแต่วันที่ 4 มิถุนายน พ.ศ.2562 จนถึง 29 พฤศจิกายน พ.ศ.2562 รวมเป็นระยะเวลาประมาณ 6 เดือน โดยในการปฏิบัติงานต่าง ๆ ในช่วงสหกิจศึกษา มีรายละเอียดดังต่อไป

\section{ตำแหน่ง/หน้าที่ของงานที่ได้รับมอบหมาย}
ปฏิบัติงานด้วยตำแหน่ง Software Engineer (Backend) ทำหน้าที่รับผิดชอบในการพัฒนาและดูแลเซิร์ฟเวอร์ของเว็บไซต์ wongnai.com เพื่อให้ผู้ใช้งานทุกแพลตฟอร์มทั้ง  Web และ Mobile Application สามารถทำงานร่วมกันได้อย่างมีประสิทธิภาพ, ควบคุมคุณภาพของโค้ดให้มีคุณภาพที่ดี, ทำงานได้ถูกต้อง, ทดสอบและดูแลได้ง่าย, มีความยืดหยุ่นพร้อมรับการเปลี่ยนแปลงในอนาคต

\section{รายละเอียดของโครงงานที่รับผิดชอบ}
โครงงานที่รับผิดชอบคือ ระบบจัดการโฆษณาแบบจำกัดจำนวนการคลิกและการแสดงโฆษณา แต่เดิมนั้นลูกค้าจะสามารถลงโฆษณาร้านของตนเองกับทาง Wongnai ได้แค่ตามช่วงเวลาที่ตกลงกันไว้ ยกตัวอย่างเช่น ลูกค้ามาขอติดต่อลงโฆษณาเป็นระยะเวลา 30 วัน ตั้งแต่วันที่ 1 กันยนยน จนถึง 30 กันยายน เป็นต้น ระบบที่พัฒนาขึ้นใหม่นั้นจะทำให้ลูกค้าสามารถลงโฆษณากับทาง Wongnai อีกรูปแบบหนึ่งโดยการลงโฆษณาแบบจำกัดจำนวนการแสดงโฆษณาและการคลิก เช่น หากลงโฆษณาไว้แล้วแสดงโฆษณาเกิน 10,000 ครั้ง หรือมีผู้ที่คลิกเข้าไปในโฆษณาครบ 5,000 คน ระบบก็จะหยุดแสดงโฆษณานั้นโดยอัตโนมัติ และยังมีระบบที่สามารถรายงานโฆษณาที่ลูกค้าลงไว้ได้อย่างอัตโนมัติอีกด้วย

\section{รายละเอียดของงานที่ปฏิบัติ}
ในการปฏิบัติงาน ได้มีการนำเทคโนโลยีและเครื่องมือต่าง ๆ มาใช้งาน เพื่อให้การพัฒนาระบบเป็นไปอย่างราบรื่น, รวดเร็ว และสามารถทำงานร่วมสมาชิกทีมคนอื่น ๆ ได้อย่างมีประสิทธิภาพ  โดยเทคโนโลยีและเครื่องมือต่าง ๆ นั้น ได้แก่
\begin{enumerate}
	\item IntelliJ IDEA
	
	IntelliJ IDEA เป็น Integrate Development Environment (IDE) สำหรับใช้ในการพัฒนาซอฟต์แวร์ที่ใช้ Java Virtual Machine (JVM) โดยเฉพาะ มีระบบ Suggestion และ Auto Completion ที่ทำให้การทำงานเป็นไปอย่างราบรื่นและรวดเร็ว ~\cite{intellij}
	
	\item Sequel Pro
	
	Sequel Pro เป็นแอปพลิเคชันสำหรับจัดการฐานข้อมูล MySQL ~\cite{sequelpro}
	
	\item Postman
	
	Postman เป็นแอปพลิเคชันสำหรับสร้าง API Request เช่น REST, SOAP, GraphQL เพื่อทดสอบการทำงาน API ของ Server และสามารถตรวจสอบ Response ที่ส่งกลับมาได้ ~\cite{postman}
	
	\item Visual Studio Code
	
	Visual Studio Code เป็น Text Editor ที่รองรับได้หลากหลายภาษา มีระบบ Syntax Highlighting ในการตรวจสอบ Syntax ของโค้ด และสามารถติดตั้งส่วนขยายต่าง ๆ เพิ่มเติมได้ตามความเหมาะสมในการทำงาน ~\cite{vscode}
	
	\item GitKraken
	
	GitKraken เป็น Git GUI Client ที่ทำให้สามารถใช้งาน Git ได้อย่างสะดวกสบาย ~\cite{gitkraken}
	
	\item Asana
	
	Asana คือระบบออนไลน์ที่คอยแสดง Workflow ของสมาชิกทีม ทำให้สมาชิกคนอื่น ๆ ในทีมสามารถทราบสถานะงานของแต่ละคนได้อย่างรวดเร็ว ~\cite{asana}
	
	\item Java
	
	Java เป็นภาษาคอมพิวเตอร์ประเภท Object-Oriented ที่สามารถนำ bytecode ที่ได้จากการคอมไพล์ ไปใช้งานบนคอมพิวเตอร์เครื่องไหนก็ได้ที่มี Java Virtual Machine (JVM) ~\cite{java}
	
	\item Python
	
	Python เป็นภาษาคอมพิวเตอร์ระดับสูงที่ใช้ Python Interpreter มีจุดเด่นที่สามารถอ่านและทำความเข้าใจโค้ดได้ง่าย โดย Python Interpreter นั้น สามารถติดตั้งได้ในหลากหลายระบบปฏิบัติการ  ~\cite{python}
	
	\item MySQL
	
	MySQL เป็นตัวจัดการฐานข้อมูลแบบ Relational (Relational Database Management: RDBMS) ที่เป็น Open source ~\cite{mysql}
	
	\item Google BigQuery
	
	Google BigQuery เป็น Data Warehouse บน Cloud ที่ให้บริการโดย Google และสามารถใช้ SQL เพื่อใช้งาน Google BigQuery ได้ ~\cite{bigquery}
	
	\item Git
	
	Git คือ Version Control ที่สามารถติดตามและควบคุมการเปลี่ยนแปลงของโค้ดได้ เพื่อให้ Software Engineer คนอื่น ๆ สามารถทำงานร่วมกันได้อย่างมีประสิทธิภาพ ~\cite{git}
	
	\item Spring Framework
	
	Spring Framework คือ เฟรมเวิร์คสำหรับพัฒนา REST API, Websocket, Web และอื่น ๆ ของภาษาที่ใช้ Java Virtual Machine (JVM) ~\cite{spring}
	
	\item Docker
	
	Docker คือ เทคโนโลยีสำหรับสร้าง Container ของซอฟต์แวร์ ทำให้ซอฟต์แวร์สามารถนำไปใช้งานในสภาพแวดล้อมไหนก็ได้ ~\cite{docker}
	
	\item Kubernetes

	Kubernetes คือ เทคโนโลยีสำหรับจัดการ Cluster (กลุ่มเครื่อง Server) สามารถจัดการ Container ที่กำลังรันระบบเพื่อให้สามารถทำงานได้อย่างต่อเนื่อง มี Downtime เป็นศูนย์ ~\cite{kubernetes}
	
	\item Gitlab CI/CD
	
	Gitlab CI/CD คือ เครื่องมือในการ Build ซอฟต์แวร์และ Deploy โดยอัตโนมัติ ~\cite{gitlabcicd}
	
\end{enumerate}

ระบบจัดการโฆษณาแบบจำกัดจำนวนการคลิกและการแสดงโฆษณาที่พัฒนาขึ้นมานั้น ประกอบด้วยเซอร์วิสต่าง ๆ ได้แก่

\begin{enumerate}
	\item Wongnai Core
	
	Wongnai Core เป็นระบบ Backend หลักของ Wongnai ถูกพัฒนาด้วยภาษา Java ทำหน้าที่ให้บริการหลาย ๆ อย่าง โดยหน้าที่ของ Wongnai Core ที่เกี่ยวข้องกับระบบจัดการโฆษณาแบบจำกัดจำนวนการคลิกและการแสดงโฆษณา ได้แก่ 
	\begin{itemize}
		\item[-] ควบคุมการแสดงผลโฆษณาของเว็บไซต์ wongnai.com และแอปพลิเคชัน Wongnai
		\item[-] ประมวลผลเมื่อได้รับข้อมูลจำนวนผู้ชมและผู้ที่คลิกเข้าไปในโฆษณาล่าสุดจาก Updater ใน Analytics Data Pipeline เพื่อนำไปพิจารณาต่อว่าควรจะนำโฆษณาที่แสดงอยู่ออกหรือไม่ โดยดูจากจำนวนผู้ชมและผู้ที่คลิกเข้าไปดูโฆษณาทั้งหมดว่าเกินกว่าตัวเลขที่จำกัดไว้ตามที่ตกลงกันไว้หรือไม่ ถ้าเกินแล้วก็จะหยุดการแสดงผลของโฆษณานั้น ๆ
		\item[-] รอรับการร้องขอข้อมูลจากระบบรายงานผลการโฆษณา เพื่อนำข้อมูลไปใช้ในการสร้างรายงานที่สมบูรณ์ส่งกลับไปยังเจ้าของโฆษณา
	\end{itemize}
	
	\item Analytics Data Pipeline
	
	Analytics Data Pipeline เป็นระบบขนาดเล็กที่ถูกพัฒนาด้วยภาษา Python ทำหน้าที่รวบรวมข้อมูลส่วนที่ต้องการจากตารางข้อมูลขนาดใหญ่ตารางเดียวใน Google BigQuery โดยตารางข้อมูลขนาดใหญ่ดังกล่าวนั้น จะจัดเก็บข้อมูลที่ฝั่งไคลเอนต์ส่งมาบันทึกไว้ ซึ่งจะส่งข้อมูลเกี่ยวกับการกระทำต่าง ๆ ที่เกิดขึ้นบนฝั่งไคลเอนต์ โดยข้อมูลที่เกี่ยวข้องกับระบบจัดการโฆษณาแบบจำกัดจำนวนการคลิกและการแสดงโฆษณา จะเป็นข้อมูลที่เกี่ยวกับการแสดงผลและการคลิกเข้าไปในโฆษณา
	
	เช่น ตำแหน่งที่รูปภาพในระบบถูกนำไปแสดงบนไคลเอนต์แพลตฟอร์ม, ตำแหน่งต่าง ๆ บนเว็บไซต์ที่มีผู้ใช้คลิกลงไป เป็นต้น หน้าที่ที่เกี่ยวข้องกับระบบจัดการโฆษณาแบบจำกัดจำนวนการคลิกและการแสดงโฆษณาสำหรับ Analytics Data Pipeline คือการรวบรวมข้อมูลส่วนที่เป็นข้อมูลการแสดงโฆษณาและข้อมูลผู้ใช้งานที่คลิกเข้าไปที่โฆษณา ไปจัดเก็บแยกไว้ในอีกตารางหนึ่งใน Google BigQuery เพื่อให้สะดวกต่อการนำข้อมูลไปใช้งานต่อ
	\begin{itemize}
		\item[-]  
		\item[-]
	\end///{itemize}
	
	\item Analytics Data Updater
	
	Analytics Data Updater เป็นระบบขนาดเล็กที่พัฒนาด้วยภาษา Python ทำหน้าที่นำข้อมูลการแสดงโฆษณาและข้อมูลผู้ใช้งานที่คลิกเข้าไปที่โฆษณาที่ Analytics Data Pipeline รวบรวมเป็นตารางขนาดเล็กไว้ให้แล้ว ส่งไปอัปเดตที่ Wongnai Core เพื่อให้ Wongnai Core ทำการประมวลผลเพื่อพิจารณาดูว่าควรจะเอาโฆษณาออกแล้วหรือไม่
	
	\item Advertisement Report Service
	
	เป็นระบบที่ถูกพัฒนาด้วยภาษา Java ทำหน้าที่สร้างรายงานของโฆษณาที่มีข้อมูลเกี่ยวกับสถิติต่าง ๆ ของโฆษณา ได้แก่ จำนวนผู้ชมโฆษณาต่อวัน, จำนวนผู้ที่คลิกเข้าไปในโฆษณาต่อวัน, จำนวนการคลิกและการชมที่ยังคงเหลืออยู่จากที่ตกลงกันไว้
\end{enumerate}

กระบวนการทำงานของระบบจัดการโฆษณาแบบจำกัดจำนวนการคลิกและการแสดงโฆษณา จะเป็นไปตามแผนผังดังต่อไปนี้

\section{ลักษณะขั้นตอนการทำงาน}
ทีม Development ของบริษัท วงใน มีเดีย จำกัด (สำนักงานใหญ่) จะถูกแบ่งออกเป็นทีมย่อย ๆ ตามประเภทของฟังก์ชันที่รับผิดชอบ เรียกว่า Squad ซึ่งจะเป็นทีมแบบ Cross-Functional กล่าวคือ ภายในทีมจะประกอบไปด้วยหลาย ๆ ฝ่าย ได้แก่ Project Manager, Software Engineer (Frontend), Software Engineer (Backend), Software Engineer (iOS), Software Engineer (Android) และ Quality Assurance Engineer

แต่ละ Squad นั้นจะทำงานโดยใช้ Scrum เป็นหลัก Scrum จะทำงานเป็นวงรอบ (Sprint) แต่ละรอบนั้นจะเท่ากับ 1 สัปดาห์ แต่ภายหลังได้มีการปรับเปลี่ยนไปเป็นรอบละ 2 สัปดาห์ โดยจะกิจกรรมที่สำคัญต่าง ๆ ดังต่อไปนี้
\begin{enumerate}
	\item Sprint Planning
	
	เป็นการประชุมตอนต้นรอบ เพื่อรับมอบหมายงานจาก Project Manager และเป็นการประชุมเพื่อปรึกษาหาวิธีการทำงานและวิธีการแก้ไขปัญหาต่าง ๆ ที่เกี่ยวกับงานที่ได้รับมอบหมาย
	
	\item Daily Meeting
	
	เป็นการประชุมแบบสั้น ๆ ประจำวัน มีจุดประสงค์เพื่อให้สมาชิกทีมรับทราบความคืบหน้าของงานที่แต่ละคนกำลังทำอยู่และทราบปัญหาที่เกิดขึ้นระหว่างการทำงาน
	
	\item Backlog Refinement Meeting
	
	เป็นการประชุมตอนกลางรอบ เพื่อพิจารณาว่างานที่ได้รับมอบหมายมา มีขนาดใหญ่หรือเล็กกว่าที่วางแผนไว้หรือไม่ และปรึกษาหาทางแก้ไขที่เหมาะสมกับสถานการณ์
	
	\item Retrospective Meeting
	
	เป็นการประชุมตอนปลายรอบ เพื่อสรุปการทำงานที่ได้ทำไปในรอบ และให้สมาชิกภายในทีมอธิบายปัญหาที่เกิดขึ้นในรอบ รวมไปถึงเรื่องราวดี ๆ ที่เกิดขึ้นในรอบด้วย เพื่อนำไปปรับปรุงการทำงานในรอบถัดไป
\end{enumerate}

การติดต่อสื่อสารภายในองค์กรจะใช้ Slack เป็นหลัก สถานะของงานภายในทีมสามารถดูได้จาก Kanban Board ซึ่งเป็นบอร์ดที่ตั้งอยู่ในพื้นที่ทำงาน และ Asana ซึ่งเป็นระบบออนไลน์ที่จะทำให้สมาชิกภายในทีมสามารถทราบสถานะของงานได้อย่างรวดเร็ว ภายในกระบวนการทำงาน สถานะของงานจะเป็นไปตามดังต่อไปนี้

\begin{enumerate}
	\item To do
	
	งานที่ยังไม่ได้เริ่มทำ แต่อยู่ในรอบแล้วจะมีสถานะเป็น To do
	
	\item In progress
	
	งานที่กำลังทำอยู่จะมีสถานะเป็น In progress
	
	\item Review
	
	เมื่องานที่ทำอยู่เสร็จแล้ว ก่อนที่จะนำงานส่วนที่ทำเข้าไปในระบบ Beta ซึ่งเป็นระบบที่มีไว้ทดสอบก่อนที่จะใช้งานจริง โค้ดที่เขียนขึ้นมาจะต้องผ่านการตรวจสอบจาก Software Engineer คนอื่นอย่างน้อย 2 คนก่อน จึงจะสามารถส่งไปให้ Quality Assurance Engineer ทำการทดสอบต่อได้
	
	\item Review passed
	
	เมื่องานที่ทำอยู่ผ่านการตรวจสอบโดย Software Engineer คนอื่นครบ 2 คนแล้ว งานจะอยู่ในสถานะ Review passed 
	
	\item Testing
	งานที่อยู่ในสถานะ Review passed จะถูกส่งต่อให้ Quality Assurance Engineer ทดสอบ ซึ่งก่อนที่จะให้ Quality Assurance Engineer ทดสอบนั้น จะต้องเตรียมวิธีการทดสอบและเตรียมข้อมูลให้เรียบร้อยก่อน
	
	\item Test passed
	
	เมื่อ Quality Assurance Engineer ทดสอบเสร็จแล้ว งานจะอยู่ในสถานะ Test passed สามารถนำงานเข้าระบบ Beta ได้เลย
	
	\item Done
	
	เมื่อนำงานขึ้นระบบ Beta เสร็จแล้ว งานจะมีสถานะเป็น Done แต่อย่างไรก็ตาม เจ้าของงานจะต้องติดตามงานของตัวเองจนกว่างานจะขึ้นอยู่บนระบบที่ใช้งานจริง
\end{enumerate}

โดยส่วนมากแล้ว ถ้าเป็นงานที่เป็นการเขียนโค้ด จะมีกระบวนตามที่กล่าวมา แต่อย่างไรก็ตามงานบางชนิดไม่จำเป็นต้องทำตามกระบวนการอย่างเคร่งครัดก็ได้ ขึ้นอยู่กับความเหมาะสมของงานว่าควรจะเป็นแบบไหน เช่น งานบางชิ้นที่เป็นการสร้างเครื่องมือให้กับพนักงานฝ่ายอื่นในบริษัทใช้ เราสามารถให้พนักงานฝ่ายนั้น ซึ่งเป็นผู้ใช้งานโดยตรงเป็นผู้ทดสอบงานของเราแทนก็ได้ เพื่อที่จะนำความคิดเห็นไปปรับปรุงงานได้อย่างตรงจุดที่สุด

การทำงานของทีม Development ที่เป็นการเขียนโค้ดจะใช้ Test Driven Development (TDD) เป็นหลัก เป็นการเขียนชุดทดสอบของโค้ดขึ้นมาก่อน แล้วรันชุดสอบให้เกิดข้อผิดพลาด จากนั้นจึงเขียนโค้ดเพื่อแก้ไขไม่ให้เกิดข้อผิดพลาดนั้น ระหว่างการเขียนโค้ดจะต้องคอยคำนึงถึงคุณภาพของโค้ด หากมีโค้ดส่วนที่ทำงานซ้ำกันจะต้องทำการ Refactoring โค้ดส่วนนั้นด้วย

\begin{figure}[!h]
	\centering
	\includegraphics[width=0.8\textwidth]{kanban-board.png}  
	\caption{Kanban Board ที่ตั้งอยู่ในพื้นที่ทำงาน}
	\label{Fig:kanban-board}
\end{figure}

\begin{figure}[!h]
	\centering
	\includegraphics[width=1\textwidth]{asana.png}  
	\caption{ตัวอย่างของโปรแกรม Asana}
	\label{Fig:asana}
\end{figure}
    \chapter{สรุปผลการปฏิบัติงาน}
\label{chapter:summary-result}

 จากการปฏิบัติงานสหกิจศึกษาที่ บริษัท วงใน มีเดีย จำกัด (สำนักงานใหญ่) ด้วยตำแหน่ง Software Engineer (Back-end) เป็นระยะเวลา 6 เดือน ตั้ง 4 มิถุนายน พ.ศ.2562 จนถึง 29 พฤศจิกายน พ.ศ.2562 สามารถสรุปผลการปฏิบัติงานได้ดังนี้
 
 \section{ผลการปฏิบัติงาน}
ระบบจัดการโฆษณาแบบจำกัดจำนวนการคลิกและการแสดงโฆษณา นอกจากจะสามารถจำกัดการแสดงผลโฆษณาด้วยจำนวนการคลิกของโฆษณา กับสามารถสร้างอีเมลรายงานสถิติโฆษณาส่งไปยังลูกค้าได้โดยอัตโนมัติแล้ว จะมีหน้าแอดมินสำหรับให้เจ้าหน้าที่ที่เกี่ยวข้องเข้ามาใช้งาน Ad Report ได้ โดยจะมีฟังก์ชันต่าง ๆ ดังนี้
\begin{itemize}
	\item แก้ไขข้อมูลต่าง ๆ ของร้านได้โดยการกดไปที่ไอคอนดินสอสีฟ้า
	\item ส่งอีเมลรายงานสถิติของโฆษณารายสัปดาห์โดยการกดไปที่ปุ่ม ACTIONS สีแดง  (สำหรับใช้งานในกรณีที่การส่งอัตโนมัติเกิดข้อผิดพลาด เจ้าหน้าที่คนอื่นจะสามารถส่งอีเมลรายงานด้วยตนเองได้) 
\end{itemize}

\begin{figure}[!h]
	\centering
	\includegraphics[width=1\textwidth]{admin-ui.png}  
	\caption{หน้าแอดมินสำหรับให้เจ้าหน้าที่ที่เกี่ยวข้องเข้ามาใช้งาน Ad Report}
	\label{Fig:admin-ui}
\end{figure}

สำหรับหน้าแอดมินได้ใช้ Framework ที่จัดเตรียมไว้ให้อยู่แล้ว พัฒนาโดยทีม Software Engineer (Frontend) ของบริษัท ใช้ React ซึ่งเป็น Library สำหรับสร้าง User Interface ของเว็บไซต์ด้วยภาษา Javascript ~\cite{react} และได้สร้างอีเมลรายงานสถิติของโฆษณาตามที่ UX/UI ของ Squad เป็นผู้ออกแบบ

\begin{figure}[!h]
	\centering
	\includegraphics[width=0.8\textwidth]{report-email.png}  
	\caption{อีเมลรายงานสถิติของโฆษณาที่ส่งให้ลูกค้า}
	\label{Fig:report-email}
\end{figure}

\begin{figure}[!h]
	\centering
	\includegraphics[width=0.525\textwidth]{report.png}  
	\caption{รายงานสถิติของโฆษณาที่ส่งให้ลูกค้า}
	\label{Fig:report}
\end{figure}

ภายในรายงานจะประกอบไปได้ส่วนต่าง ๆ ได้แก่

\begin{itemize}
	\item ชื่อร้าน
	\item ช่วงเวลาของรายงาน
	\item จำนวนครั้งที่แสดงผลโฆษณาในช่วงเวลาของรายงาน
	\item จำนวนครั้งที่มีผู้ใช้คลิกเข้าไปที่โฆษณาในช่วงเวลาของรายงาน
	\item แผนภูมิแสดงจำนวนครั้งที่แสดงผลโฆษณาในช่วงเวลาของรายงานต่อวัน
	\item แผนภูมิแสดงจำนวนครั้งที่มีผู้ใช้คลิกเข้าไปที่โฆษณาในช่วงเวลาของรายงานต่อวัน
	\item จำนวนคลิกของโฆษณาที่ใช้ไปแล้ว
	\item จำนวนคลิกของโฆษณาคงเหลือ
\end{itemize}

\section{ประโยชน์ที่ได้รับจากการปฏิบัติงาน}
\begin{enumerate}
	\item ประโยชน์ต่อตนเอง
	\begin{itemize}
		\item ได้รับความรู้และเทคนิคต่าง ๆ เกี่ยวกับการสร้างซอฟต์แวร์ ทั้งวิธีการแก้ปัญหาต่าง ๆ และวิธีการสร้างซอฟต์แวร์ให้มีคุณภาพ Software Engineer คนอื่นสามารถทำความเข้าใจ, แก้ไข และพัฒนาซอฟต์แวร์ต่อได้ง่าย	
		\item ได้รับประสบการณ์จากการทำงานจริง, ฝึกฝนการทำงานภายใต้แรงกดดันและเวลาที่จำกัด และฝึกฝนการสื่อสารกับสมาชิกทีมและภายในองค์กร
		\item ได้รับแนวคิดใหม่ ๆ ทั้งการทำงานและอื่น ๆ ที่เป็นประโยชน์ในอนาคต
	\end{itemize}
	\item ประโยชน์ต่อสถานประกอบการ
	\begin{itemize}
		\item สร้างระบบใหม่เพื่อเพิ่มประสิทธิภาพในการหารายได้ขององค์กร
		\item ช่วยลดภาระของพนักงานประจำ เพื่อให้พนักงานประจำสามารถจดจ่อกับการทำงานหลักได้อย่างเต็มที่
	\end{itemize}
	\item ประโยชน์ต่อมหาวิทยาลัย
	\begin{itemize}
		\item ได้รับความไว้วางใจและการยอมรับจากสถานประกอบการ
		\item ได้รับข้อมูลเพื่อนำไปปรับปรุงกระบวนการเรียนการสอน เพื่อให้นักศึกษามีศักยภาพที่ตรงกับความต้องการในตลาด
	\end{itemize}
\end{enumerate}

\section{วิเคราะห์จุดเด่น จุดด้อย โอกาส อุปสรรค (SWOT Analysis)}
\begin{enumerate}
	\item จุดเด่น
	\begin{itemize}
		\item ตั้งใจทำงานอย่างเต็มที่ เพื่อให้ผลงานออกมาดีที่สุด
	\end{itemize}
	\item จุดด้อย
	\begin{itemize}
		\item ยังขาดทักษะในการสื่อสาร ทำให้เกิดการเข้าใจไม่ตรงกัน
		\item ยังขาดทักษะในการทำงาน ทำให้งานเกิดความล่าช้า
	\end{itemize}
	\item โอกาส
	\begin{itemize}
		\item ได้มีส่วนร่วมในการพัฒนาระบบให้กับบริษัทใหญ่
		\item ได้เรียนรู้ความรู้และวิธีการใหม่ ๆ เพื่อพัฒนาความสามารถของตนเอง
		\item ได้รับการช่วยเหลือจากพนักงานหลาย ๆ ท่าน ทำให้การทำงานเป็นไปได้อย่างราบรื่น
	\end{itemize}
	\item อุปสรรค
	\begin{itemize}
		\item เนื่องจากยังขาดทักษะในการสื่อสาร ทำให้การทำงานบางจุดเป็นไปอย่างยากลำบาก
		\item การขาดทักษะในการทำงานที่ดี ทำให้งานบางจุดทำได้อย่างล่าช้า
	\end{itemize}
\end{enumerate}

\section{ประสบการณ์ที่ประทับใจ / ประสบการณ์พิเศษ}
การทำงานในช่วงเวลาปกติ พนักงานทุก ๆ ท่าน ให้ความช่วยเหลือและให้ความร่วมมือกันเป็นอย่างดีเวลาที่เกิดปัญหาใด ๆ ก็สามารถถามพนักงานได้ทันที ทำให้งานสามารถดำเนินไปได้อย่างราบรื่น ถึงแม้จะเป็นงานที่ยากก็ตาม ได้เรียนรู้สิ่งที่เป็นประโยชน์หลายอย่าง ไม่ว่าจะเป็นทักษะในการทำงาน ทักษะในการสื่อสาร หรือเรื่องทั่วไปอื่น ๆ

ตลอดระยะเวลาสหกิจศึกษาที่บริษัท วงใน มีเดีย จำกัด (สำนักงานใหญ่) จะมีกิจกรรมต่าง ๆ อยู่ตลอดเวลา ทั้งกิจกรรมเพื่อการเรียนรู้และกิจกรรมเพื่อความสนุกสนาน

    \chapter{ปัญหาและข้อเสนอแนะ}
\label{chapter:problems-and-solution}

จากการปฏิบัติงานสหกิจศึกษาที่ บริษัท วงใน มีเดีย จำกัด (สำนักงานใหญ่) ด้วยตำแหน่ง Software Engineer (Back-end) เป็นระยะเวลา 6 เดือน ตั้ง 4 มิถุนายน พ.ศ.2562 จนถึง 29 พฤศจิกายน พ.ศ.2562 ได้พบกับปัญหาหลายประการ ดังนี้

\begin{enumerate}
	\item ปัญหาด้านสถานประกอบการ
	
	เนื่องจากสถานประกอบการเป็นสถานประกอบการขนาดใหญ่ งานส่วนมากจะเป็นการดูแลรักษาระบบเดิมที่มีอยู่มากกว่าการพัฒนาระบบใหม่ ทำให้งานที่นักศึกษาได้รับ อาจจะไม่ตรงกับความต้องการของสหกิจศึกษาที่ต้องการให้งานออกมาในรูปแบบโครงงาน
	
	ข้อเสนอแนะหรือแนวทางการแก้ไข – สถานประกอบการควรเตรียมงานให้กับนักศึกษาก่อนที่จะถึงช่วงสหกิจศึกษา
	
	\item ปัญหาด้านมหาวิทยาลัย
	
	มหาวิทยาลัยมีตัวเลือกบริษัทและตำแหน่งในสหกิจศึกษาน้อย ไม่ตรงกับความต้องการของนักศึกษา, การดำเนินการเรื่องเอกสารเป็นไปอย่างล่าช้า และการแจ้งข้อมูลต่าง ๆ กับนักศึกษาและสถานประกอบการยังคงเป็นไปอย่างล่าช้า
	
	ข้อเสนอแนะหรือแนวทางการแก้ไข – มหาวิทยาลัยควรมีตัวเลือกบริษัทและตำแหน่งในสหกิจศึกษาให้มากกว่านี้ และควรปรับปรุงการดำเนินการเรื่องเอกสารกับแจ้งข่าวสารให้รวดเร็วกว่านี้
	
	\item ปัญหาด้านตัวนักศึกษา
	
	นักศึกษายังขาดทักษะในการทำงานที่ดีและขาดทักษะการสื่อสารในการทำงาน ทำให้งานดำเนินไปอย่างล่าช้า และมีโอกาสผิดพลาดสูง
	
	ข้อเสนอแนะหรือแนวทางการแก้ไข – นักศึกษาควรปรับตัวให้เข้ากับสถานประกอบการเร็วกว่านี้ และควรฝึกฝนทักษะการสื่อสารและทักษะการทำงานให้มากกว่านี้
\end{enumerate}
    
    \clearpage
    \addcontentsline{toc}{chapter}{บรรณานุกรม}
    \bibliographystyle{IEEEtran}
    \bibliography{reference}
    
    \startappendix
    \chapter{สถานที่ปฏิบัติงาน}

\chapter{กิจกรรมระหว่างปฏิบัติงาน}

\chapter{ประวัติผู้เขียน}
\AddToShipoutPicture*
{\put(470,700){\includegraphics[width=3cm,height=3.1cm]{me.png}}}
\begin{tabularx}{\linewidth}{lX}
	\textbf{ชื่อ – นามสกุล} & มาวิน จงไกรรัตนกุล \\
	\textbf{Email} & mw.jkrtnk@gmail.com \\
	\textbf{ประวัติการศึกษา} & วิทยาศาสตร์​บัณฑิต สาขาวิชาเทคโนโลยีสารสนเทศ \\ & คณะเทคโนโลยีสารสนเทศ สถาบันพระจอมเกล้าเจ้าคุณทหารลาดกระบัง \\
\end{tabularx}
    
\end{document}