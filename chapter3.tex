\chapter{สรุปผลการปฏิบัติงาน}
\label{chapter:summary-result}

 จากการปฏิบัติงานสหกิจศึกษาที่ บริษัท วงใน มีเดีย จำกัด (สำนักงานใหญ่) ด้วยตำแหน่ง Software Engineer (Back-end) เป็นระยะเวลา 6 เดือน ตั้ง 4 มิถุนายน พ.ศ.2562 จนถึง 29 พฤศจิกายน พ.ศ.2562 สามารถสรุปผลการปฏิบัติงานได้ดังนี้
 
 \section{ผลการปฏิบัติงาน}
ระบบจัดการโฆษณาแบบจำกัดจำนวนการคลิกและการแสดงโฆษณา นอกจากจะสามารถจำกัดการแสดงผลโฆษณาด้วยจำนวนการคลิกของโฆษณา กับสามารถสร้างอีเมลรายงานสถิติโฆษณาส่งไปยังลูกค้าได้โดยอัตโนมัติแล้ว จะมีหน้าแอดมินสำหรับให้เจ้าหน้าที่ที่เกี่ยวข้องเข้ามาใช้งาน Ad Report ได้ โดยจะมีฟังก์ชันต่าง ๆ ดังนี้
\begin{itemize}
	\item แก้ไขข้อมูลต่าง ๆ ของร้านได้โดยการกดไปที่ไอคอนดินสอสีฟ้า
	\item ส่งอีเมลรายงานสถิติของโฆษณารายสัปดาห์โดยการกดไปที่ปุ่ม ACTIONS สีแดง  (สำหรับใช้งานในกรณีที่การส่งอัตโนมัติเกิดข้อผิดพลาด เจ้าหน้าที่คนอื่นจะสามารถส่งอีเมลรายงานด้วยตนเองได้) 
\end{itemize}

\begin{figure}[!h]
	\centering
	\includegraphics[width=1\textwidth]{admin-ui.png}  
	\caption{หน้าแอดมินสำหรับให้เจ้าหน้าที่ที่เกี่ยวข้องเข้ามาใช้งาน Ad Report}
	\label{Fig:admin-ui}
\end{figure}

สำหรับหน้าแอดมินได้ใช้ Framework ที่จัดเตรียมไว้ให้อยู่แล้ว พัฒนาโดยทีม Software Engineer (Frontend) ของบริษัท ใช้ React ซึ่งเป็น Library สำหรับสร้าง User Interface ของเว็บไซต์ด้วยภาษา Javascript ~\cite{react} และได้สร้างอีเมลรายงานสถิติของโฆษณาตามที่ UX/UI ของ Squad เป็นผู้ออกแบบ

\begin{figure}[!h]
	\centering
	\includegraphics[width=0.8\textwidth]{report-email.png}  
	\caption{อีเมลรายงานสถิติของโฆษณาที่ส่งให้ลูกค้า}
	\label{Fig:report-email}
\end{figure}

\begin{figure}[!h]
	\centering
	\includegraphics[width=0.525\textwidth]{report.png}  
	\caption{รายงานสถิติของโฆษณาที่ส่งให้ลูกค้า}
	\label{Fig:report}
\end{figure}

ภายในรายงานจะประกอบไปได้ส่วนต่าง ๆ ได้แก่

\begin{itemize}
	\item ชื่อร้าน
	\item ช่วงเวลาของรายงาน
	\item จำนวนครั้งที่แสดงผลโฆษณาในช่วงเวลาของรายงาน
	\item จำนวนครั้งที่มีผู้ใช้คลิกเข้าไปที่โฆษณาในช่วงเวลาของรายงาน
	\item แผนภูมิแสดงจำนวนครั้งที่แสดงผลโฆษณาในช่วงเวลาของรายงานต่อวัน
	\item แผนภูมิแสดงจำนวนครั้งที่มีผู้ใช้คลิกเข้าไปที่โฆษณาในช่วงเวลาของรายงานต่อวัน
	\item จำนวนคลิกของโฆษณาที่ใช้ไปแล้ว
	\item จำนวนคลิกของโฆษณาคงเหลือ
\end{itemize}

\section{ประโยชน์ที่ได้รับจากการปฏิบัติงาน}
\begin{enumerate}
	\item ประโยชน์ต่อตนเอง
	\begin{itemize}
		\item ได้รับความรู้และเทคนิคต่าง ๆ เกี่ยวกับการสร้างซอฟต์แวร์ ทั้งวิธีการแก้ปัญหาต่าง ๆ และวิธีการสร้างซอฟต์แวร์ให้มีคุณภาพ Software Engineer คนอื่นสามารถทำความเข้าใจ, แก้ไข และพัฒนาซอฟต์แวร์ต่อได้ง่าย	
		\item ได้รับประสบการณ์จากการทำงานจริง, ฝึกฝนการทำงานภายใต้แรงกดดันและเวลาที่จำกัด และฝึกฝนการสื่อสารกับสมาชิกทีมและภายในองค์กร
		\item ได้รับแนวคิดใหม่ ๆ ทั้งการทำงานและอื่น ๆ ที่เป็นประโยชน์ในอนาคต
	\end{itemize}
	\item ประโยชน์ต่อสถานประกอบการ
	\begin{itemize}
		\item สร้างระบบใหม่เพื่อเพิ่มประสิทธิภาพในการหารายได้ขององค์กร
		\item ช่วยลดภาระของพนักงานประจำ เพื่อให้พนักงานประจำสามารถจดจ่อกับการทำงานหลักได้อย่างเต็มที่
	\end{itemize}
	\item ประโยชน์ต่อมหาวิทยาลัย
	\begin{itemize}
		\item ได้รับความไว้วางใจและการยอมรับจากสถานประกอบการ
		\item ได้รับข้อมูลเพื่อนำไปปรับปรุงกระบวนการเรียนการสอน เพื่อให้นักศึกษามีศักยภาพที่ตรงกับความต้องการในตลาด
	\end{itemize}
\end{enumerate}

\section{วิเคราะห์จุดเด่น จุดด้อย โอกาส อุปสรรค (SWOT Analysis)}
\begin{enumerate}
	\item จุดเด่น
	\begin{itemize}
		\item ตั้งใจทำงานอย่างเต็มที่ เพื่อให้ผลงานออกมาดีที่สุด
	\end{itemize}
	\item จุดด้อย
	\begin{itemize}
		\item ยังขาดทักษะในการสื่อสาร ทำให้เกิดการเข้าใจไม่ตรงกัน
		\item ยังขาดทักษะในการทำงาน ทำให้งานเกิดความล่าช้า
	\end{itemize}
	\item โอกาส
	\begin{itemize}
		\item ได้มีส่วนร่วมในการพัฒนาระบบให้กับบริษัทใหญ่
		\item ได้เรียนรู้ความรู้และวิธีการใหม่ ๆ เพื่อพัฒนาความสามารถของตนเอง
		\item ได้รับการช่วยเหลือจากพนักงานหลาย ๆ ท่าน ทำให้การทำงานเป็นไปได้อย่างราบรื่น
	\end{itemize}
	\item อุปสรรค
	\begin{itemize}
		\item เนื่องจากยังขาดทักษะในการสื่อสาร ทำให้การทำงานบางจุดเป็นไปอย่างยากลำบาก
		\item การขาดทักษะในการทำงานที่ดี ทำให้งานบางจุดทำได้อย่างล่าช้า
	\end{itemize}
\end{enumerate}

\section{ประสบการณ์ที่ประทับใจ / ประสบการณ์พิเศษ}
การทำงานในช่วงเวลาปกติ พนักงานทุก ๆ ท่าน ให้ความช่วยเหลือและให้ความร่วมมือกันเป็นอย่างดีเวลาที่เกิดปัญหาใด ๆ ก็สามารถถามพนักงานได้ทันที ทำให้งานสามารถดำเนินไปได้อย่างราบรื่น ถึงแม้จะเป็นงานที่ยากก็ตาม ได้เรียนรู้สิ่งที่เป็นประโยชน์หลายอย่าง ไม่ว่าจะเป็นทักษะในการทำงาน ทักษะในการสื่อสาร หรือเรื่องทั่วไปอื่น ๆ

ตลอดระยะเวลาสหกิจศึกษาที่บริษัท วงใน มีเดีย จำกัด (สำนักงานใหญ่) จะมีกิจกรรมต่าง ๆ อยู่ตลอดเวลา ทั้งกิจกรรมเพื่อการเรียนรู้และกิจกรรมเพื่อความสนุกสนาน
