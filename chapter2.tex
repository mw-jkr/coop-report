\chapter{รายละเอียดการปฏิบัติงาน}
\label{chapter:related-theory}

เริ่มสหกิจศึกษาโดยปฏิบัติงานที่ บริษัท วงใน มีเดีย จำกัด (สำนักงานใหญ่) ตั้งแต่วันที่ 4 มิถุนายน พ.ศ.2562 จนถึง 29 พฤศจิกายน พ.ศ.2562 รวมเป็นระยะเวลาประมาณ 6 เดือน โดยในการปฏิบัติงานต่าง ๆ ในช่วงสหกิจศึกษา มีรายละเอียดดังต่อไป

\section{ตำแหน่ง/หน้าที่ของงานที่ได้รับมอบหมาย}
ปฏิบัติงานด้วยตำแหน่ง Software Engineer (Back-end) ทำหน้าที่รับผิดชอบในการสร้างและดูแลระบบ Backend ของเว็บไซต์ wongnai.com เพื่อให้ผู้ใช้งานในทุกแพลตฟอร์ม เช่น Web และ Mobile Application สามารถทำงานร่วมกันได้อย่างมีประสิทธิภาพ, ควบคุมคุณภาพของโค้ดให้มีคุณภาพที่ดี ทำงานได้ถูกต้อง ทดสอบและดูแลได้ง่าย มีความยืดหยุ่นพร้อมรับการเปลี่ยนแปลงในอนาคต

\section{รายละเอียดของโครงงานที่รับผิดชอบ}
โครงงานที่รับผิดชอบคือ ระบบจัดการโฆษณาแบบจำกัดจำนวนการคลิกและการแสดงโฆษณา เมื่อก่อนลูกค้าจะสามารถลงโฆษณาร้านของตนเองกับทาง Wongnai ได้ แค่ตามช่วงเวลาที่ตกลงกันไว้ ยกตัวอย่างเช่น ลูกค้ามาขอติดต่อลงโฆษณา 30 วัน ตั้งแต่วันที่ 1 กันยนยน จนถึง 30 กันยายน เป็นต้น ระบบที่พัฒนาขึ้นใหม่นั้นจะทำให้ลูกค้าสามารถลงโฆษณากับทาง Wongnai ในอีกรูปแบบหนึ่งได้ โดยการลงโฆษณาแบบจำกัดจำนวนการแสดงโฆษณาและการคลิก เช่น หากลงโฆษณาไว้แล้วแสดงโฆษณาเกิน 10,000 ครั้ง หรือมีผู้ที่คลิกเข้าไปในโฆษณาครบ 5,000 คน ระบบก็จะนำโฆษณาออกโดยอัตโนมัติ อีกทั้งยังมีระบบที่สามารถรายงานสถานะของโฆษณาที่ลงไว้ได้ โดยภายในรายงานจะประกอบไปด้วยรายละเอียดเกี่ยวกับสถิติของโฆษณา ได้แก่ จำนวนผู้ชมโฆษณาต่อวัน, จำนวนผู้ที่คลิกเข้าไปในโฆษณา รวมไปถึงจำนวนการดูและการคลิกที่คงเหลือจากที่ตกลงกันไว้

\section{รายละเอียดของงานที่ปฏิบัติ}
ในการปฏิบัติงาน นอกจากจะมีการนำเทคโนโลยีและเครื่องมือต่าง ๆ ที่จำเป็นมาใช้งานแล้ว ก็จะมีใช้งานเทคโนโลยีและเครื่องมืออื่น ๆ เพื่อให้การพัฒนาซอฟต์แวร์เป็นไปอย่างราบรื่นและรวดเร็ว และสามารถทำงานร่วมสมาชิกทีมคนอื่น ๆ ได้อย่างมีประสิทธิภาพ  โดยเทคโนโลยีและเครื่องมือต่าง ๆ นั้น ได้แก่
\begin{enumerate}
	\item IntelliJ IDEA
	
	IntelliJ IDEA เป็น Integrate Development Environment (IDE) สำหรับใช้ในการพัฒนาซอฟต์แวร์ที่ใช้ Java Virtual Machine (JVM) โดยเฉพาะ มีระบบ Suggestion และ Auto Completion ที่ทำให้การทำงานเป็นไปอย่างราบรื่นและรวดเร็ว
	
	\item Sequel Pro
	
	Sequel Pro เป็นแอปพลิเคชันสำหรับจัดการฐานข้อมูล MySQL
	
	\item Postman
	
	Postman เป็นแอปพลิเคชันสำหรับสร้าง API Request เช่น REST, SOAP, GraphQL เพื่อทดสอบการทำงาน API ของ Server และสามารถตรวจสอบ Response ที่ส่งกลับมาได้
	
	\item Visual Studio Code
	
	Visual Studio Code เป็น Text Editor ที่รองรับได้หลากหลายภาษา มีระบบ Syntax Highlighting ในการตรวจสอบ Syntax ของโค้ด, มีระบบ Auto Completion และสามารถติดตั้งส่วนขยายต่าง ๆ เพิ่มเติมได้ ตามความเหมาะสมในการทำงาน
	
	\item Java
	
	Java เป็นภาษาคอมพิวเตอร์ประเภท Object-Oriented ที่สามารถนำ bytecode ที่ได้จากการ Compile ไปใช้งานบนคอมพิวเตอร์เครื่องไหนก็ได้ที่มี Java Virtual Machine (JVM)
	
	\item Python
	
	Python เป็นภาษาคอมพิวเตอร์ระดับสูงที่ใช้ Python Interpreter มีจุดเด่นที่สามารถอ่านและทำความเข้าใจโค้ดได้ง่าย โดย Python Interpreter นั้น สามารถติดตั้งได้ในหลากหลายระบบปฏิบัติการ 
	
	\item MySQL
	
	MySQL เป็นตัวจัดการฐานข้อมูลแบบ Relational (Relational Database Management: RDBMS) ที่เป็น Open source
	
	\item Google BigQuery
	
	Google BigQuery เป็น Data Warehouse บน Cloud ที่ให้บริการโดย Google และสามารถใช้ SQL เพื่อใช้งานได้ ทุก ๆ ครั้งที่ดึงข้อมูลจาก Google BigQuery เราจะต้องเสียเครดิตตามขนาดของข้อมูลที่ดึงมา
	
	\item Git
	
	Git คือ Version Control ที่สามารถติดตามและควบคุมการเปลี่ยนแปลงของโค้ดได้ เพื่อให้ Software Engineer คนอื่น ๆ สามารถทำงานร่วมกันได้อย่างมีประสิทธิภาพ
	
	\item GitKraken
	
	GitKraken เป็น Git GUI Client ที่ทำให้สามารถใช้งาน Git ได้อย่างสะดวกสบาย
	
	\item Asana
	
	Asana คือระบบออนไลน์ที่คอยแสดง Workflow ของทีม ทำให้สมาชิกคนอื่น ๆ ในทีมสามารถทราบสถานะงานของแต่ละคนได้อย่างรวดเร็ว
\end{enumerate}

ระบบจัดการโฆษณาแบบจำกัดจำนวนการคลิกและการแสดงโฆษณาที่พัฒนาขึ้นมานั้น ประกอบด้วยเซอร์วิสต่าง ๆ ได้แก่

\begin{enumerate}
	\item Wongnai Core
	
	Wongnai Core เป็นระบบ Back-end หลักของ Wongnai ถูกพัฒนาด้วยภาษา Java ทำหน้าที่ให้บริการหลาย ๆ อย่าง โดยหน้าที่ของ Wongnai Core ที่เกี่ยวข้องกับระบบจัดการโฆษณาแบบจำกัดจำนวนการคลิกและการแสดงโฆษณา คือการควบคุมการแสดงผลโฆษณาของเว็บไซต์ wongnai.com และแอปพลิเคชัน Wongnai และคอยประมวลผลเมื่อได้รับข้อมูลจำนวนผู้ชมและผู้ที่คลิกเข้าไปในโฆษณาล่าสุดจาก Analytics Data Updater เพื่อนำไปพิจารณาต่อว่าควรจะนำโฆษณาที่แสดงอยู่ออกหรือไม่ โดยดูจากจำนวนผู้ชมและผู้ที่คลิกเข้าไปดูโฆษณาทั้งหมดว่าเกินกว่าตัวเลขที่จำกัดไว้ตามที่ตกลงกันไว้หรือไม่ ถ้าเกินแล้วก็จะหยุดการแสดงผลของโฆษณานั้น ๆ
	
	\item Analytics Data Pipeline
	
	Analytics Data Pipeline เป็นระบบขนาดเล็กที่ถูกพัฒนาด้วยภาษา Python ทำหน้าที่รวบรวมข้อมูลส่วนที่เราต้องการจากตารางข้อมูลขนาดใหญ่ตารางเดียวใน Google BigQuery โดยตารางข้อมูลขนาดใหญ่ดังกล่าวนั้น จะจัดเก็บข้อมูลที่ฝั่งไคลเอนต์ส่งมาบันทึกไว้ ซึ่งจะส่งข้อมูลเกี่ยวกับการกระทำต่าง ๆ ที่เกิดขึ้นบนฝั่งไคลเอนต์ เช่น ตำแหน่งที่รูปภาพในระบบถูกนำไปแสดงบนไคลเอนต์แพลตฟอร์ม, ตำแหน่งต่าง ๆ บนเว็บไซต์ที่มีผู้ใช้คลิกลงไป เป็นต้น หน้าที่ที่เกี่ยวข้องกับระบบจัดการโฆษณาแบบจำกัดจำนวนการคลิกและการแสดงโฆษณาสำหรับ Analytics Data Pipeline คือการรวบรวมข้อมูลส่วนที่เป็นข้อมูลการแสดงโฆษณาและข้อมูลผู้ใช้งานที่คลิกเข้าไปที่โฆษณา ไปจัดเก็บแยกไว้ในอีกตารางหนึ่งใน Google BigQuery เพื่อให้สะดวกต่อการนำข้อมูลไปใช้งานต่อ
	
	\item Analytics Data Updater
	
	Analytics Data Updater เป็นระบบขนาดเล็กที่พัฒนาด้วยภาษา Python ทำหน้าที่นำข้อมูลการแสดงโฆษณาและข้อมูลผู้ใช้งานที่คลิกเข้าไปที่โฆษณาที่ Analytics Data Pipeline รวบรวมเป็นตารางขนาดเล็กไว้ให้แล้ว ส่งไปอัปเดตที่ Wongnai Core เพื่อให้ Wongnai Core ทำการประมวลผลเพื่อพิจารณาดูว่าควรจะเอาโฆษณาออกแล้วหรือไม่
	
	\item Advertisement Report Service
	
	เป็นระบบที่ถูกพัฒนาด้วยภาษา Java ทำหน้าที่สร้างรายงานของโฆษณาที่มีข้อมูลเกี่ยวกับสถิติต่าง ๆ ของโฆษณา ได้แก่ จำนวนผู้ชมโฆษณาต่อวัน, จำนวนผู้ที่คลิกเข้าไปในโฆษณาต่อวัน, จำนวนการคลิกและการชมที่ยังคงเหลืออยู่จากที่ตกลงกันไว้
\end{enumerate}

กระบวนการทำงานของระบบจัดการโฆษณาแบบจำกัดจำนวนการคลิกและการแสดงโฆษณา จะเป็นไปตามแผนผังดังต่อไปนี้