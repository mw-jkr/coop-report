\chapter{รายละเอียดการปฏิบัติงาน}
\label{chapter:related-theory}

เริ่มสหกิจศึกษาโดยปฏิบัติงานที่ บริษัท วงใน มีเดีย จำกัด (สำนักงานใหญ่) ตั้งแต่วันที่ 4 มิถุนายน พ.ศ.2562 จนถึง 29 พฤศจิกายน พ.ศ.2562 รวมเป็นระยะเวลาประมาณ 6 เดือน โดยในการปฏิบัติงานต่าง ๆ ในช่วงสหกิจศึกษา มีรายละเอียดดังต่อไป

\section{ตำแหน่ง/หน้าที่ของงานที่ได้รับมอบหมาย}
ปฏิบัติงานด้วยตำแหน่ง Software Engineer (Backend) ทำหน้าที่รับผิดชอบในการพัฒนาและดูแลเซิร์ฟเวอร์ของเว็บไซต์ wongnai.com เพื่อให้ผู้ใช้งานทุกแพลตฟอร์มทั้ง  Web และ Mobile Application สามารถทำงานร่วมกันได้อย่างมีประสิทธิภาพ, ควบคุมคุณภาพของโค้ดให้มีคุณภาพที่ดี, ทำงานได้ถูกต้อง, ทดสอบและดูแลได้ง่าย, มีความยืดหยุ่นพร้อมรับการเปลี่ยนแปลงในอนาคต

\section{รายละเอียดของโครงงานที่รับผิดชอบ}
โครงงานที่รับผิดชอบคือ ระบบจัดการโฆษณาแบบจำกัดจำนวนการคลิกและการแสดงโฆษณา แต่เดิมนั้นลูกค้าจะสามารถลงโฆษณาร้านของตนเองกับทาง Wongnai ได้แค่ตามช่วงเวลาที่ตกลงกันไว้ ยกตัวอย่างเช่น ลูกค้ามาขอติดต่อลงโฆษณาเป็นระยะเวลา 30 วัน ตั้งแต่วันที่ 1 กันยนยน จนถึง 30 กันยายน เป็นต้น ระบบที่พัฒนาขึ้นใหม่นั้นจะทำให้ลูกค้าสามารถลงโฆษณากับทาง Wongnai อีกรูปแบบหนึ่งโดยการลงโฆษณาแบบจำกัดจำนวนการแสดงโฆษณาและการคลิก เช่น หากลงโฆษณาไว้แล้วแสดงโฆษณาเกิน 10,000 ครั้ง หรือมีผู้ที่คลิกเข้าไปในโฆษณาครบ 5,000 คน ระบบก็จะหยุดแสดงโฆษณานั้นโดยอัตโนมัติ และยังมีระบบที่สามารถรายงานโฆษณาที่ลูกค้าลงไว้ได้อย่างอัตโนมัติอีกด้วย

\section{รายละเอียดของงานที่ปฏิบัติ}
ในการปฏิบัติงาน ได้มีการนำเทคโนโลยีและเครื่องมือต่าง ๆ มาใช้งาน เพื่อให้การพัฒนาระบบเป็นไปอย่างราบรื่น, รวดเร็ว และสามารถทำงานร่วมสมาชิกทีมคนอื่น ๆ ได้อย่างมีประสิทธิภาพ  โดยเทคโนโลยีและเครื่องมือต่าง ๆ นั้น ได้แก่
\begin{enumerate}
	\item IntelliJ IDEA
	
	IntelliJ IDEA เป็น Integrate Development Environment (IDE) สำหรับใช้ในการพัฒนาซอฟต์แวร์ที่ใช้ Java Virtual Machine (JVM) โดยเฉพาะ มีระบบ Suggestion และ Auto Completion ที่ทำให้การทำงานเป็นไปอย่างราบรื่นและรวดเร็ว ~\cite{intellij}
	
	\item Sequel Pro
	
	Sequel Pro เป็นแอปพลิเคชันสำหรับจัดการฐานข้อมูล MySQL ~\cite{sequelpro}
	
	\item Postman
	
	Postman เป็นแอปพลิเคชันสำหรับสร้าง API Request เช่น REST, SOAP, GraphQL เพื่อทดสอบการทำงาน API ของ Server และสามารถตรวจสอบ Response ที่ส่งกลับมาได้ ~\cite{postman}
	
	\item Visual Studio Code
	
	Visual Studio Code เป็น Text Editor ที่รองรับได้หลากหลายภาษา มีระบบ Syntax Highlighting ในการตรวจสอบ Syntax ของโค้ด และสามารถติดตั้งส่วนขยายต่าง ๆ เพิ่มเติมได้ตามความเหมาะสมในการทำงาน ~\cite{vscode}
	
	\item GitKraken
	
	GitKraken เป็น Git GUI Client ที่ทำให้สามารถใช้งาน Git ได้อย่างสะดวกสบาย ~\cite{gitkraken}
	
	\item Asana
	
	Asana คือระบบออนไลน์ที่คอยแสดง Workflow ของสมาชิกทีม ทำให้สมาชิกคนอื่น ๆ ในทีมสามารถทราบสถานะงานของแต่ละคนได้อย่างรวดเร็ว ~\cite{asana}
	
	\item Java
	
	Java เป็นภาษาคอมพิวเตอร์ประเภท Object-Oriented ที่สามารถนำ bytecode ที่ได้จากการคอมไพล์ ไปใช้งานบนคอมพิวเตอร์เครื่องไหนก็ได้ที่มี Java Virtual Machine (JVM) ~\cite{java}
	
	\item Python
	
	Python เป็นภาษาคอมพิวเตอร์ระดับสูงที่ใช้ Python Interpreter มีจุดเด่นที่สามารถอ่านและทำความเข้าใจโค้ดได้ง่าย โดย Python Interpreter นั้น สามารถติดตั้งได้ในหลากหลายระบบปฏิบัติการ  ~\cite{python}
	
	\item MySQL
	
	MySQL เป็นตัวจัดการฐานข้อมูลแบบ Relational (Relational Database Management: RDBMS) ที่เป็น Open source ~\cite{mysql}
	
	\item Google BigQuery
	
	Google BigQuery เป็น Data Warehouse บน Cloud ที่ให้บริการโดย Google และสามารถใช้ SQL เพื่อใช้งาน Google BigQuery ได้ ~\cite{bigquery}
	
	\item Git
	
	Git คือ Version Control ที่สามารถติดตามและควบคุมการเปลี่ยนแปลงของโค้ดได้ เพื่อให้ Software Engineer คนอื่น ๆ สามารถทำงานร่วมกันได้อย่างมีประสิทธิภาพ ~\cite{git}
	
	\item Spring Framework
	
	Spring Framework คือ เฟรมเวิร์คสำหรับพัฒนา REST API, Websocket, Web และอื่น ๆ ของภาษาที่ใช้ Java Virtual Machine (JVM) ~\cite{spring}
	
	\item Docker
	
	Docker คือ เทคโนโลยีสำหรับสร้าง Container ของซอฟต์แวร์ ทำให้ซอฟต์แวร์สามารถนำไปใช้งานในสภาพแวดล้อมไหนก็ได้ ~\cite{docker}
	
	\item Kubernetes

	Kubernetes คือ เทคโนโลยีสำหรับจัดการ Cluster (กลุ่มเครื่อง Server) สามารถจัดการ Container ที่กำลังรันระบบเพื่อให้สามารถทำงานได้อย่างต่อเนื่อง มี Downtime เป็นศูนย์ ~\cite{kubernetes}
	
	\item Gitlab CI/CD
	
	Gitlab CI/CD คือ เครื่องมือในการ Build ซอฟต์แวร์และ Deploy โดยอัตโนมัติ ~\cite{gitlabcicd}
	
\end{enumerate}

ระบบจัดการโฆษณาแบบจำกัดจำนวนการคลิกและการแสดงโฆษณาที่พัฒนาขึ้นมานั้น ประกอบด้วยเซอร์วิสต่าง ๆ ได้แก่

\begin{enumerate}
	\item Wongnai Core
	
	Wongnai Core เป็นระบบ Backend หลักของ Wongnai ถูกพัฒนาด้วยภาษา Java ทำหน้าที่ให้บริการหลาย ๆ อย่าง โดยหน้าที่ของ Wongnai Core ที่เกี่ยวข้องกับระบบจัดการโฆษณาแบบจำกัดจำนวนการคลิกและการแสดงโฆษณา ได้แก่ 
	\begin{itemize}
		\item[-] ควบคุมการแสดงผลโฆษณาของเว็บไซต์ wongnai.com และแอปพลิเคชัน Wongnai
		\item[-] ประมวลผลเมื่อได้รับข้อมูลจำนวนผู้ชมและผู้ที่คลิกเข้าไปในโฆษณาล่าสุดจาก Updater ใน Analytics Data Pipeline เพื่อนำไปพิจารณาต่อว่าควรจะนำโฆษณาที่แสดงอยู่ออกหรือไม่ โดยดูจากจำนวนผู้ชมและผู้ที่คลิกเข้าไปดูโฆษณาทั้งหมดว่าเกินกว่าตัวเลขที่จำกัดไว้ตามที่ตกลงกันไว้หรือไม่ ถ้าเกินแล้วก็จะหยุดการแสดงผลของโฆษณานั้น ๆ
		\item[-] รอรับการร้องขอข้อมูลจากระบบรายงานผลการโฆษณา เพื่อนำข้อมูลไปใช้ในการสร้างรายงานที่สมบูรณ์ส่งกลับไปยังเจ้าของโฆษณา
	\end{itemize}
	
	\item Analytics Data Pipeline
	
	Analytics Data Pipeline เป็นระบบขนาดเล็กที่ถูกพัฒนาด้วยภาษา Python ทำหน้าที่รวบรวมข้อมูลส่วนที่ต้องการจากตารางข้อมูลขนาดใหญ่ตารางเดียวใน Google BigQuery โดยตารางข้อมูลขนาดใหญ่ดังกล่าวนั้น จะจัดเก็บข้อมูลที่ฝั่งไคลเอนต์ส่งมาบันทึกไว้ ซึ่งจะส่งข้อมูลเกี่ยวกับการกระทำต่าง ๆ ที่เกิดขึ้นบนฝั่งไคลเอนต์ โดยข้อมูลที่เกี่ยวข้องกับระบบจัดการโฆษณาแบบจำกัดจำนวนการคลิกและการแสดงโฆษณา จะเป็นข้อมูลที่เกี่ยวกับการแสดงผลและการคลิกเข้าไปในโฆษณา
	
	เช่น ตำแหน่งที่รูปภาพในระบบถูกนำไปแสดงบนไคลเอนต์แพลตฟอร์ม, ตำแหน่งต่าง ๆ บนเว็บไซต์ที่มีผู้ใช้คลิกลงไป เป็นต้น หน้าที่ที่เกี่ยวข้องกับระบบจัดการโฆษณาแบบจำกัดจำนวนการคลิกและการแสดงโฆษณาสำหรับ Analytics Data Pipeline คือการรวบรวมข้อมูลส่วนที่เป็นข้อมูลการแสดงโฆษณาและข้อมูลผู้ใช้งานที่คลิกเข้าไปที่โฆษณา ไปจัดเก็บแยกไว้ในอีกตารางหนึ่งใน Google BigQuery เพื่อให้สะดวกต่อการนำข้อมูลไปใช้งานต่อ
	\begin{itemize}
		\item[-]  
		\item[-]
	\end///{itemize}
	
	\item Analytics Data Updater
	
	Analytics Data Updater เป็นระบบขนาดเล็กที่พัฒนาด้วยภาษา Python ทำหน้าที่นำข้อมูลการแสดงโฆษณาและข้อมูลผู้ใช้งานที่คลิกเข้าไปที่โฆษณาที่ Analytics Data Pipeline รวบรวมเป็นตารางขนาดเล็กไว้ให้แล้ว ส่งไปอัปเดตที่ Wongnai Core เพื่อให้ Wongnai Core ทำการประมวลผลเพื่อพิจารณาดูว่าควรจะเอาโฆษณาออกแล้วหรือไม่
	
	\item Advertisement Report Service
	
	เป็นระบบที่ถูกพัฒนาด้วยภาษา Java ทำหน้าที่สร้างรายงานของโฆษณาที่มีข้อมูลเกี่ยวกับสถิติต่าง ๆ ของโฆษณา ได้แก่ จำนวนผู้ชมโฆษณาต่อวัน, จำนวนผู้ที่คลิกเข้าไปในโฆษณาต่อวัน, จำนวนการคลิกและการชมที่ยังคงเหลืออยู่จากที่ตกลงกันไว้
\end{enumerate}

กระบวนการทำงานของระบบจัดการโฆษณาแบบจำกัดจำนวนการคลิกและการแสดงโฆษณา จะเป็นไปตามแผนผังดังต่อไปนี้

\section{ลักษณะขั้นตอนการทำงาน}
ทีม Development ของบริษัท วงใน มีเดีย จำกัด (สำนักงานใหญ่) จะถูกแบ่งออกเป็นทีมย่อย ๆ ตามประเภทของฟังก์ชันที่รับผิดชอบ เรียกว่า Squad ซึ่งจะเป็นทีมแบบ Cross-Functional กล่าวคือ ภายในทีมจะประกอบไปด้วยหลาย ๆ ฝ่าย ได้แก่ Project Manager, Software Engineer (Frontend), Software Engineer (Backend), Software Engineer (iOS), Software Engineer (Android) และ Quality Assurance Engineer

แต่ละ Squad นั้นจะทำงานโดยใช้ Scrum เป็นหลัก Scrum จะทำงานเป็นวงรอบ (Sprint) แต่ละรอบนั้นจะเท่ากับ 1 สัปดาห์ แต่ภายหลังได้มีการปรับเปลี่ยนไปเป็นรอบละ 2 สัปดาห์ โดยจะกิจกรรมที่สำคัญต่าง ๆ ดังต่อไปนี้
\begin{enumerate}
	\item Sprint Planning
	
	เป็นการประชุมตอนต้นรอบ เพื่อรับมอบหมายงานจาก Project Manager และเป็นการประชุมเพื่อปรึกษาหาวิธีการทำงานและวิธีการแก้ไขปัญหาต่าง ๆ ที่เกี่ยวกับงานที่ได้รับมอบหมาย
	
	\item Daily Meeting
	
	เป็นการประชุมแบบสั้น ๆ ประจำวัน มีจุดประสงค์เพื่อให้สมาชิกทีมรับทราบความคืบหน้าของงานที่แต่ละคนกำลังทำอยู่และทราบปัญหาที่เกิดขึ้นระหว่างการทำงาน
	
	\item Backlog Refinement Meeting
	
	เป็นการประชุมตอนกลางรอบ เพื่อพิจารณาว่างานที่ได้รับมอบหมายมา มีขนาดใหญ่หรือเล็กกว่าที่วางแผนไว้หรือไม่ และปรึกษาหาทางแก้ไขที่เหมาะสมกับสถานการณ์
	
	\item Retrospective Meeting
	
	เป็นการประชุมตอนปลายรอบ เพื่อสรุปการทำงานที่ได้ทำไปในรอบ และให้สมาชิกภายในทีมอธิบายปัญหาที่เกิดขึ้นในรอบ รวมไปถึงเรื่องราวดี ๆ ที่เกิดขึ้นในรอบด้วย เพื่อนำไปปรับปรุงการทำงานในรอบถัดไป
\end{enumerate}

การติดต่อสื่อสารภายในองค์กรจะใช้ Slack เป็นหลัก สถานะของงานภายในทีมสามารถดูได้จาก Kanban Board ซึ่งเป็นบอร์ดที่ตั้งอยู่ในพื้นที่ทำงาน และ Asana ซึ่งเป็นระบบออนไลน์ที่จะทำให้สมาชิกภายในทีมสามารถทราบสถานะของงานได้อย่างรวดเร็ว ภายในกระบวนการทำงาน สถานะของงานจะเป็นไปตามดังต่อไปนี้

\begin{enumerate}
	\item To do
	
	งานที่ยังไม่ได้เริ่มทำ แต่อยู่ในรอบแล้วจะมีสถานะเป็น To do
	
	\item In progress
	
	งานที่กำลังทำอยู่จะมีสถานะเป็น In progress
	
	\item Review
	
	เมื่องานที่ทำอยู่เสร็จแล้ว ก่อนที่จะนำงานส่วนที่ทำเข้าไปในระบบ Beta ซึ่งเป็นระบบที่มีไว้ทดสอบก่อนที่จะใช้งานจริง โค้ดที่เขียนขึ้นมาจะต้องผ่านการตรวจสอบจาก Software Engineer คนอื่นอย่างน้อย 2 คนก่อน จึงจะสามารถส่งไปให้ Quality Assurance Engineer ทำการทดสอบต่อได้
	
	\item Review passed
	
	เมื่องานที่ทำอยู่ผ่านการตรวจสอบโดย Software Engineer คนอื่นครบ 2 คนแล้ว งานจะอยู่ในสถานะ Review passed 
	
	\item Testing
	งานที่อยู่ในสถานะ Review passed จะถูกส่งต่อให้ Quality Assurance Engineer ทดสอบ ซึ่งก่อนที่จะให้ Quality Assurance Engineer ทดสอบนั้น จะต้องเตรียมวิธีการทดสอบและเตรียมข้อมูลให้เรียบร้อยก่อน
	
	\item Test passed
	
	เมื่อ Quality Assurance Engineer ทดสอบเสร็จแล้ว งานจะอยู่ในสถานะ Test passed สามารถนำงานเข้าระบบ Beta ได้เลย
	
	\item Done
	
	เมื่อนำงานขึ้นระบบ Beta เสร็จแล้ว งานจะมีสถานะเป็น Done แต่อย่างไรก็ตาม เจ้าของงานจะต้องติดตามงานของตัวเองจนกว่างานจะขึ้นอยู่บนระบบที่ใช้งานจริง
\end{enumerate}

โดยส่วนมากแล้ว ถ้าเป็นงานที่เป็นการเขียนโค้ด จะมีกระบวนตามที่กล่าวมา แต่อย่างไรก็ตามงานบางชนิดไม่จำเป็นต้องทำตามกระบวนการอย่างเคร่งครัดก็ได้ ขึ้นอยู่กับความเหมาะสมของงานว่าควรจะเป็นแบบไหน เช่น งานบางชิ้นที่เป็นการสร้างเครื่องมือให้กับพนักงานฝ่ายอื่นในบริษัทใช้ เราสามารถให้พนักงานฝ่ายนั้น ซึ่งเป็นผู้ใช้งานโดยตรงเป็นผู้ทดสอบงานของเราแทนก็ได้ เพื่อที่จะนำความคิดเห็นไปปรับปรุงงานได้อย่างตรงจุดที่สุด

การทำงานของทีม Development ที่เป็นการเขียนโค้ดจะใช้ Test Driven Development (TDD) เป็นหลัก เป็นการเขียนชุดทดสอบของโค้ดขึ้นมาก่อน แล้วรันชุดสอบให้เกิดข้อผิดพลาด จากนั้นจึงเขียนโค้ดเพื่อแก้ไขไม่ให้เกิดข้อผิดพลาดนั้น ระหว่างการเขียนโค้ดจะต้องคอยคำนึงถึงคุณภาพของโค้ด หากมีโค้ดส่วนที่ทำงานซ้ำกันจะต้องทำการ Refactoring โค้ดส่วนนั้นด้วย

\begin{figure}[!h]
	\centering
	\includegraphics[width=0.8\textwidth]{kanban-board.png}  
	\caption{Kanban Board ที่ตั้งอยู่ในพื้นที่ทำงาน}
	\label{Fig:kanban-board}
\end{figure}

\begin{figure}[!h]
	\centering
	\includegraphics[width=1\textwidth]{asana.png}  
	\caption{ตัวอย่างของโปรแกรม Asana}
	\label{Fig:asana}
\end{figure}